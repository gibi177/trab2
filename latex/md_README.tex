Trabalho 2 da disciplina de Métodos de Programação\+: Problema das 8 rainhas

Consiste em receber um arquivo de texto com um tabuleiro de xadrez na forma de uma matriz 8x8. O número 0 indica que há um espaço vazio e o 1 indica que há uma rainha naquela posição.

Uma entrada é solução se há é uma matriz 8x8, tem exatamente 8 rainhas no tabuleiro, as entradas da matriz são 0 ou 1 e se nenhuma da 8 rainhas do tabuleiro se ataca.

Na saída, 1 indica que a entrada é solução do problema, 0 indica que não é solução e -\/1 indica que a entrada é inválida.

As entradas estão em arquivos .txt na pasta \char`\"{}entrada\char`\"{}, caso a entrada não seja solução, será criado um arquivo na pasta \char`\"{}ataques\char`\"{} mostrando qual par de rainhas se ataca.

Está escrito em C++ e inclui testes abrangentes usando a framework de testes Catch2.

Requisitos\+: Um compilador C++, aqui foi usado o g++; Framework de testes Catch2.

Instruções\+: Crie um arquivo para o makefile e copie-\/o para o seu projeto. Com todos os programas instalados, digite \char`\"{}make\char`\"{} no terminal para compilar e rodar os testes. ~\newline
 